Setup R\+OS catkin workspace, {\ttfamily git clone} to {\ttfamily catkin\+\_\+ws/src/} and build the catkin workspace.

{\ttfamily poetry shell} 
\begin{DoxyCode}
source ../../devel/setup.bash
cd rostest
rosrun rostest main.py
\end{DoxyCode}


\subsection*{Docker}

Building the image

{\ttfamily sudo docker build -\/-\/tag rostest\+:1.\+0 .}

Create network for communicating with roscore

{\ttfamily sudo docker network create rosnet}

Start up roscore 
\begin{DoxyCode}
sudo docker run -it --rm \(\backslash\)
--net rosnet \(\backslash\)
--name master \(\backslash\)
ros:melodic-ros-core \(\backslash\)
roscore
\end{DoxyCode}
 Start up rostest node in another console 
\begin{DoxyCode}
sudo docker run -it --rm \(\backslash\)
    --net rosnet \(\backslash\)
    --name rostest \(\backslash\)
    --env ROS\_HOSTNAME=rostest \(\backslash\)
    --env ROS\_MASTER\_URI=http://master:11311 \(\backslash\)
    rostest:1.0
\end{DoxyCode}
 Attaching camera devices to the container can be done by adding parameters to node container startup

{\ttfamily -\/-\/device} argument, for example {\ttfamily -\/-\/device /dev/video0}

\subsubsection*{Docker compose}

Alternatively the above can be done with docker compose, with the following on the root directory of the project.

{\ttfamily docker-\/compose build}

{\ttfamily docker-\/compose up}

For sending messages to rostest.\+py attach to input container with another terminal\+:

{\ttfamily sudo docker attach rosinput}

For seeing recieved messages in separate terminal use\+:

{\ttfamily sudo docker attach rostest}

For seeing recieved information about the objects use\+:

{\ttfamily sudo docker attach rosprinter}

Attaching devices like webcams \href{https://docs.docker.com/compose/compose-file/#devices}{\tt https\+://docs.\+docker.\+com/compose/compose-\/file/\#devices}

\subsection*{Versions}

Tested to work on Docker version 19.\+03.\+8-\/ce and Docker-\/compose version 1.\+25.\+5 